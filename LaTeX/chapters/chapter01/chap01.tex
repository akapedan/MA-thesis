%
% File: chap01.tex
% Author: Your Name
% Description: Introduction.
%
\let\textcircled=\pgftextcircled
\chapter{Introduction}
\label{chap:intro}
%=========================================================

\initial{W}e only have one earth, that is certain. That we also need to treat it well in order to cope with the climate crisis surrounding the human-made rise in temperature, is also certain. Such statements have also gained prominence in the financial world in recent years. Indeed, the large financial players hold a key role in ensuring whether sustainable technologies and projects receive long-term financing or whether industries that are detrimental to the climate are cut off. Shareholders and bank customers play a pivotal role in this development by shaping demand and, in the words of \citet{fama2007disagreement}, creating a "taste" for sustainability. Consequently, progress is also being made on the supply side, with a variety of so-called green products already on the market. In this context, this paper examines a particular type of debt security, namely the "green version" of a bond. These so-called green bonds differ from their conventional version in only one aspect, which is that their proceeds are earmarked for a green purpose. 

Since 2016, academics have increasingly questioned the extent to which these green bonds are profitable for issuers and investors. It is argued that green bonds fetch a so-called greenium. This effect means that investors are willing to pay a premium for the green bonds, which, ceteris paribus, lowers the interest rate, due to the negative relationship between the price and the interest rate of a bond. In a nutshell, the investor bears more compared to conventional bonds, while the issuer can finance his project more cost-effectively due to a lower interest rate. Scholars have addressed this issue, but the evidence remains inconclusive. The goal of this paper is therefore to shed new light on this green bond premium puzzle. We contribute to this literature by studying the greenium in the primary market using a novel tool from the causal machine learning literature. In particular, we apply the causal forest methodology developed in \citet{wager2018estimation}. By leveraging the causal inference framework from \citet{rubin1974estimating}, we interpret the green labeling of bonds as treatment and define our study design as selection-on-observables. In addition to gauging the average treatment effect (ATE), which we interpret as the greenium effect, causal forests are specialized in detecting heterogeneous treatment effects (HTE). Therefore, the present study also aims to analyze the possible determinants affecting the presence or absence of the greenium effect.

To answer the questions, this paper is structured as follows. Firstly, the reader is introduced thematically by outlining the underlying principles of the green bond market and the greenium. This is followed by a review of the academic literature examining the greenium in the primary market. In a second step, the data set and the methods applied are described in more depth. The description of the methodology is followed by the central part of the paper, the quantitative analysis. Therein, we verify the model assumptions, present the descriptive statistics as well as the results of the causal forest and the findings of our Monte-Carlo simulation. In Chapter 5, we reflect on the results and discuss possible limitations. We also provide an outlook on possible future research directions. Finally, we conclude this thesis in Chapter 6 with a summary of the main conclusions.