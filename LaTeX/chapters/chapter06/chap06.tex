%
% File: chap06.tex
% Author: Your Name
% Description: Conclusion
%
\let\textcircled=\pgftextcircled
\chapter{Conclusion}
\label{chap:results}

\initial{T}he present thesis examined the greenium effect in the primary bond market by applying the causal forest method of the recent causal machine learning literature. It was shown that indeed a greenium is present with an estimated value between 18.6 to 103.2 basis points. The central tendency of our sample amounts to 43 basis points. These results are statistically and economically significant. 

The process underlying our findings was as follows: We first attempted to provide a description of the green bond market and its peculiar phenomenon of the greenium. Thereafter, we conducted a thorough literature analysis in order to educate ourselves and the readers on the past research efforts which assessed the greenium in the primary market. Chapter three described the Eikon Refinitiv dataset, the variables which we extracted from it and the methodological approach. In particular, especially for robustness reasons, we divided our initial dataset in 3 subsets. Moreover, we used three different methodological specifications: (1) Estimation without issuer controls, (2) estimation with issuer controls and (3) additional propensity score matching before estimation. This approach provided us with a total of 11 models. In chapter four we verified the model assumptions as well as presented the descriptive statistics and causal forest results. For the sake of readability as well as comprehensibility, we have limited our presentation to the two most informative models, which also allows a simplified interpretation of the results based on the subsets. Besides finding negative average treatment effects throughout all of our models, we also aimed to detect possible heterogeneity. Indeed, we have found heterogeneity for which the following covariates play an substantial role with respect to the quartile with the highest greenium effect: (1) longer time to maturity, (2) issued during 2013 to 2016, (2) annual coupon, (3) not in the financial sector but rather in energy and utilities and (4) issued in euros. Another interesting result is that our CBI subset in combination with the propensity score matching method, which is our most rigorous model in terms of both the definition of "green" and the bond sample, has the highest greenium effect compared to the other subsets which used the PSM approach. Therefore, this means that the additional certification costs are passed on to the investor, who bears a lower return on investment. Finally, due to the possible violation of the i.i.d assumption of our outcome variable, i.e. yield at issuance, we provided a Monte-Carlo simulation which computed the effects of such a violation on the accuracy of the average treatment effect estimate. Our results show that depending on the magnitude of the dependence, the effect is a less precise estimate due to an inflated variance. Nevertheless, the effect is relatively small for both small and large infringements. To conclude, our study ended with a discussion of the contributions and limitations, as well as recommendations for future research.

%=======
